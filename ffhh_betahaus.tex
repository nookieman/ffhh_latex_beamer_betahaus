\documentclass[aspectratio=43]{beamer}
\usetheme{intridea}  %% Themenwahl

\usepackage[ngerman]{babel} 
\usepackage[T1]{fontenc} 				% richtige Silbentrennung
\usepackage[utf8]{inputenc}    % Umlaute etc.! 
\title{Freifunk Hamburg}
\author{Freifunk Hamburg}
\date{\today}

\begin{document}
\maketitle
\frame{\tableofcontents[currentsection]}

\section{Abschnitt 1}
\begin{frame} %%Eine Folie
  \frametitle{Was ist freifunk.net} %%Folientitel
  \begin{itemize}
	\item hamburg.freifunk ist eine nichtkommerzielle Initiative, die in Zusammenarbeit mit anderen Organisationen und Gruppen die Idee freier Netzwerke fördert
	\item \textbf{frei} wird dabei verstanden als:
	\begin{itemize}
		\item öffentlich - jede\_m zugänglich
		\item nicht kommerziell
	\end{itemize}
  \end{itemize}

\end{frame}


\begin{frame}{Welche Regeln sollte man beim Mitmachen beachten?}
	\begin{itemize}
		\item Dienstbereitstellung erfolgt ausschließlich auf freiwilliger Basis
		\item Es gibt keinerlei Nutzungsgarantie
		\item Entgelte (z.B. für Wartung/Pflege der WLAN-Router, DSL) nachbarschaftlich regeln, nie für Internetangebot kommerziell Geld verlangen
		\item Haftung - generell gilt: verantwortlich ist immer, wer die unerlaubte Handlung vollführt, nicht wer den hamburger Freifunkknoten bereitstellt
	\end{itemize}
	\it{Freifunkknoten = Freifunkrouter} %TODO raus?
\end{frame}

\end{document}